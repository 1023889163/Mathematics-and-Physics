%==============================
%!TEX program = xelatex
\documentclass[zihao=-4,a4paper]{ctexart}
\usepackage{cnote}
%==============================
\title{牛逼}
\author[$\dagger$]{\kaishu 牛逼}
\affil[$\dagger$]{\kaishu 北京大学兽医系}
\begin{document}
\maketitle
%==============================
\begin{abstract}
    装逼我让你飞起来.
\end{abstract}
%==============================
{\centering \Large\bfseries 目录 \par}
\begin{multicols}{2}
    \tableofcontents
\end{multicols}
%==============================
%==============================


%====================
\section{例子}
我们的模板支持什么:
\begin{enumerate}
	\item 物理$\ket{1}$, $\bra{1}$, $\mel**{1}{\mathbb{1}}{1}$, $\braket{1}$.
	\item \bfr{五}\bfb{颜}\bfp{六}\bfo{色}的文字
\end{enumerate}
%==========
\begin{mexample}{一些命题}
	\begin{enumerate}
		\item 雪是白的 = T
		\item 煤炭是黑的 = T
		\item 好大的雪 = 不是命题
		\item 你妈死了 = 不是命题
		\item 一个偶数可以表示成两个素数之和 - 是命题, 一定是T或者F的陈述, 但是只不过我们不知道T还是F
	\end{enumerate}
\end{mexample}
%==========
%==========
\begin{mdefinition}{总是对的或者总是错的}
	一个命题$p$如果总是正确T的, 就叫做tautology, 一个命题总是错F的叫做contradiction
\end{mdefinition}
%==========
%==========
\begin{mnote}{到底有几个算符}
	我们先来考虑一元算符, 我们的输出结果是2个, 但是能决定输出结果的输入可以是T或者F, 所以算符有$2^2=4$个.
	
	对于二元算符, 我们的输出结果还是2个, 但是能觉得输出结果的输入是TF的顺序组合, 有四个, 所以算符有$2^4=16$个!
		我们先来考虑一元算符, 我们的输出结果是2个, 但是能决定输出结果的输入可以是T或者F, 所以算符有$2^2=4$个.
	
	对于二元算符, 我们的输出结果还是2个, 但是能觉得输出结果的输入是TF的顺序组合, 有四个, 所以算符有$2^4=16$个!
		我们先来考虑一元算符, 我们的输出结果是2个, 但是能决定输出结果的输入可以是T或者F, 所以算符有$2^2=4$个.
	
	对于二元算符, 我们的输出结果还是2个, 但是能觉得输出结果的输入是TF的顺序组合, 有四个, 所以算符有$2^4=16$个!
		我们先来考虑一元算符, 我们的输出结果是2个, 但是能决定输出结果的输入可以是T或者F, 所以算符有$2^2=4$个.
	
	对于二元算符, 我们的输出结果还是2个, 但是能觉得输出结果的输入是TF的顺序组合, 有四个, 所以算符有$2^4=16$个!
		我们先来考虑一元算符, 我们的输出结果是2个, 但是能决定输出结果的输入可以是T或者F, 所以算符有$2^2=4$个.
	
	对于二元算符, 我们的输出结果还是2个, 但是能觉得输出结果的输入是TF的顺序组合, 有四个, 所以算符有$2^4=16$个!
		我们先来考虑一元算符, 我们的输出结果是2个, 但是能决定输出结果的输入可以是T或者F, 所以算符有$2^2=4$个.
	
	对于二元算符, 我们的输出结果还是2个, 但是能觉得输出结果的输入是TF的顺序组合, 有四个, 所以算符有$2^4=16$个!
		我们先来考虑一元算符, 我们的输出结果是2个, 但是能决定输出结果的输入可以是T或者F, 所以算符有$2^2=4$个.
	
	对于二元算符, 我们的输出结果还是2个, 但是能觉得输出结果的输入是TF的顺序组合, 有四个, 所以算符有$2^4=16$个!
		我们先来考虑一元算符, 我们的输出结果是2个, 但是能决定输出结果的输入可以是T或者F, 所以算符有$2^2=4$个.
	
	对于二元算符, 我们的输出结果还是2个, 但是能觉得输出结果的输入是TF的顺序组合, 有四个, 所以算符有$2^4=16$个!
		我们先来考虑一元算符, 我们的输出结果是2个, 但是能决定输出结果的输入可以是T或者F, 所以算符有$2^2=4$个.
	
	对于二元算符, 我们的输出结果还是2个, 但是能觉得输出结果的输入是TF的顺序组合, 有四个, 所以算符有$2^4=16$个!
		我们先来考虑一元算符, 我们的输出结果是2个, 但是能决定输出结果的输入可以是T或者F, 所以算符有$2^2=4$个.
	
	对于二元算符, 我们的输出结果还是2个, 但是能觉得输出结果的输入是TF的顺序组合, 有四个, 所以算符有$2^4=16$个!
\end{mnote}
%==========
%==========
\begin{mtheorem}{否定命题的导出}
		如果$p$, $q$都是命题, 那么我们必有
		\begin{equation}
			(p \Rightarrow q) \Leftrightarrow ((\neg q)\Rightarrow(\neg p))
		\end{equation}
\bfr{注意这里$p, q$的顺序!}
\end{mtheorem}
%==========
%==========
\begin{mproof}{证明否定命题的导出}
	\begin{align}
\begin{array}{c|c||c|c|c|c}
p & q & \neg p & \neg q & p \Rightarrow q & (\neg q) \Rightarrow(\neg p) \\
\hline \mathrm{F} & \mathrm{F} & \mathrm{T} & \mathrm{T} & \mathrm{T} & \mathrm{T} \\
\mathrm{F} & \mathrm{T} & \mathrm{T} & \mathrm{F} & \mathrm{T} & \mathrm{T} \\
\mathrm{T} & \mathrm{F} & \mathrm{F} & \mathrm{T} & \mathrm{F} & \mathrm{F} \\
\mathrm{T} & \mathrm{T} & \mathrm{F} & \mathrm{F} & \mathrm{T} & \mathrm{T}
\end{array}
\end{align}
我们发现最后两栏是等价的, O.K.!
\end{mproof}
%==========
数学:
\begin{equation}
	\int f(x) \dd{x} = 0
\end{equation}
%==============================
%==============================
\end{document}